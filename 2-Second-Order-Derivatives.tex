\documentclass[10pt,aspectratio=169]{beamer}
%\usepackage[T1]{fontenc} DO NOT ENABLE THIS!!!

\usetheme[progressbar=frametitle]{metropolis}
\usepackage{appendixnumberbeamer}
\setbeamertemplate{footline}{}
\setbeamertemplate{navigation symbols}[horizontal]

\setbeamertemplate{title page}{
    \begin{minipage}[c][\paperheight]{\textwidth}
        \ifx\inserttitlegraphic\@empty\else\usebeamertemplate*{title graphic}\fi
        \vfill%
        \ifx\inserttitle\@empty\else\usebeamertemplate*{title}\fi
        \ifx\insertsubtitle\@empty\else\usebeamertemplate*{subtitle}\fi
        \usebeamertemplate*{title separator}
        \begin{minipage}[t]{\textwidth}
            \ifx\beamer@shortauthor\@empty\else\usebeamertemplate*{author}\fi
            \ifx\insertdate\@empty\else\usebeamertemplate*{date}\fi
            \ifx\insertinstitute\@empty\else\usebeamertemplate*{institute}\fi
        \end{minipage}
        \vfill
        \vspace*{1mm}
    \end{minipage}
}

\metroset{sectionpage=none}
\setbeamertemplate{section in toc}[sections numbered]
\setbeamertemplate{subsection in toc}[subsections numbered]


\usepackage{booktabs}
\usepackage[scale=2]{ccicons}

\usepackage{amsmath, amsfonts}

\usepackage{float}

\usepackage{tabularx}
\usepackage{booktabs}

\usepackage{xcolor}
\definecolor{linkcolour}{rgb}{0,0.2,0.6}

\usepackage{hyperref}
\hypersetup{pdfauthor={Li, Binghuan}, 
            colorlinks,breaklinks,urlcolor=linkcolour, linkcolor=linkcolour}

\title{Approximation of the Second Derivative}
\subtitle{Finite Difference Method\cite{Ferziger2002}}
\author{Binghuan W Li \ $\lvert$ \   \href{mailto:binghuan.li19@imperial.ac.uk}{binghuan.li19@imperial.ac.uk}}
\institute{Department of Bioengineering, Imperial College London}
\titlegraphic{\includegraphics[width=0.3\textwidth]{imgs/Imperial.eps}}


%================================================
\begin{document}
\maketitle
%================================================
% \begin{frame}{Contents}
% \tableofcontents
% \end{frame}
%================================================
\begin{frame}{Approximation of the Second Derivative (1/)}
    Geometrically, the second derivative is the slope of the line tangent to the curve representing the first derivative:
    \[
    \bigg( \frac{\partial^{2}\phi}{\partial x^{2}} \bigg)_{i} \approx
    \frac{\bigg( \frac{\partial \phi}{\partial x} \bigg)_{i+1} - \bigg( \frac{\partial \phi}{\partial x} \bigg)_{i}}{x_{i+1} - x_{i}}
    \]
    In the equation above, the outer derivative was estimated by \textbf{FDS}. For inner derivatives, one may use a different approximation, for example \textbf{BDS}. This would result in the following expression:
    \begin{equation} \label{eqn:taylor}
        \bigg( \frac{\partial^{2}\phi}{\partial x^{2}} \bigg)_{i} = 
        \frac{\phi_{i+1}(x_{i}-x_{i-1}) + \phi_{i-1}(x_{i+1}-x_{i})-\phi_{i}(x_{i+1}-x_{i-1})}{(x_{i+1}-x_{i})^{2} (x_{i}-x_{i-1})}  
    \end{equation}
\end{frame}
%================================================
\begin{frame}{Approximation of the Second Derivative (2/)} \small
    One could also use \textbf{CDS} to estimate second derivative, this process require two first derivatives at $x_{i+1}$ and $x_{i-1}$. A \textit{better} choice is evaluating $\partial \phi/\partial x$ at points halfway between $x_{i}$ and $x_{i+1}$ (or $x_{i-1}$):
    \[
    \bigg( \frac{\partial\phi}{\partial x} \bigg)_{i+\frac{1}{2}} \approx \frac{\phi_{i+1}-\phi_{i}}{x_{i+1}-x_{i}}, 
    \quad \quad 
    \bigg( \frac{\partial\phi}{\partial x} \bigg)_{i-\frac{1}{2}} \approx \frac{\phi_{i}-\phi_{i-1}}{x_{i}-x_{i-1}}
    \]
    Thus, 
    \begin{align*}
       \bigg( \frac{\partial^{2}\phi}{\partial x^{2}} \bigg)_{i} & \approx 
       \frac{\bigg( \frac{\partial\phi}{\partial x} \bigg)_{i+\frac{1}{2}} - \bigg( \frac{\partial\phi}{\partial x} \bigg)_{i-\frac{1}{2}}}{\frac{1}{2}(x_{i+1}-x_{i-1})} \\
       & \approx \frac{\phi_{i+1}(x_{i}-x_{i-1}) + \phi_{i-1}(x_{i+1}-x_{i})-\phi_{i}(x_{i+1}-x_{i-1})}{\frac{1}{2}(x_{i+1}-x_{i-1}) (x_{i+1}-x_{i}) (x_{i}-x_{i-1})}
    \end{align*}
    For equidistant spacing of the points, \textit{i.e.} $\Delta x$ is constant, the expression above becomes 
    \[
    \bigg( \frac{\partial^{2}\phi}{\partial x^{2}} \bigg)_{i} \approx
    \frac{\phi_{i+1} + \phi_{i-1} -2\phi_{i}}{(\Delta x)^{2}}
    \]
\end{frame}
%================================================
\begin{frame}{Taylor Series Expansion}
Re-derive \autoref{eqn:taylor}, one can obtain an explicit expression for the error:
\begin{align*}
    \bigg( \frac{\partial^{2}\phi}{\partial x^{2}} \bigg)_{i} &=
    \frac{\phi_{i+1}(x_{i}-x_{i-1}) + \phi_{i-1}(x_{i+1}-x_{i})-\phi_{i}(x_{i+1}-x_{i-1})}{\frac{1}{2}(x_{i+1}-x_{i-1}) (x_{i+1}-x_{i}) (x_{i}-x_{i-1})}\\
    & - \frac{(x_{i+1}-x_{i}) - (x_{i}-x_{i-1})}{3} \bigg(\frac{\partial^{3}\phi}{\partial x^{3}}\bigg)_{i}+H
\end{align*}
The leading truncation error is first-order, but vanishes when the spacing between the points is uniform. If not, the truncation error is reducing in a second-order manner when the grid is refined.
\end{frame}
%================================================
\begin{frame}{Interpolation (1/)}
One can use interpolation to fit a polynomial of degree $n$ through $n+1$ data points. From that, approximations to all derivatives up to the $n$\textsuperscript{th} term can be obtained by differentiation. \newline \newline
In general, the truncation error of the approximation to the second derivative is the degree of interpolating polynomial minus one.
\begin{itemize}
    \item a polynomial of degree four fir through five points leads to a 4\textsuperscript{th}-order approximation on uniform grids:
    \[
    \bigg( \frac{\partial^{2}\phi}{\partial x^{2}} \bigg)_{i} = \frac{-\phi_{i+2}+16\phi_{i+1}-30\phi_{i}+16\phi_{i-1}-\phi_{i-1}}{12(\Delta x)^{2}} + \mathcal{O}((\Delta x)^{4})
    \]
\end{itemize}
\end{frame}
%================================================
\begin{frame}{Interpolation (2/)}
One can also use approximations of the second derivative to \textbf{improve the accuracy} of the first derivative.
\begin{itemize}
    \item First-order forward derivative: 
    \[ 
    \bigg(\frac{\partial \phi}{\partial x}\bigg)_{i} = 
    \frac{\phi_{i+1}-\phi_{i}}{x_{i+1}-x_{i}}
    - \frac{(x_{i+1}-x_{i})}{2}\bigg(\frac{\partial^{2} \phi}{\partial x^{2}}\bigg)_{i}
    {\color{gray}-\frac{(x_{i+1}-x_{i})^{2}}{6}\bigg(\frac{\partial^{3} \phi}{\partial x^{3}}\bigg)_{i} + H}
    \]
    \item Second-order CDS:
    \[
    \bigg( \frac{\partial^{2}\phi}{\partial x^{2}} \bigg)_{i} \approx
    \frac{\phi_{i+1} + \phi_{i-1} -2\phi_{i}}{(\Delta x)^{2}}
    \]
    \item These two would result in
    \[
    \bigg(\frac{\partial \phi}{\partial x}\bigg)_{i} \approx \frac{\phi_{i+1}(\Delta x_{i})^{2} - \phi_{i-1}(\Delta x_{i+1})^{2} + \phi_{i}[(\Delta x_{i+1})^{2}-(\Delta x_{i})^{2}]}{\Delta x_{i+1}\Delta x_{i} (\Delta x_{i}+\Delta x_{i+1})}
    \]
    This expression express a second order truncation error and reduce to a first-order CDS expression on uniform grids.
\end{itemize}
\end{frame}
%================================================
\begin{frame}{The Diffusive Term}
In Navier-Stokes, the conservative form of the diffusive term is often approximated using a second-order, central-difference manner:
\begin{align*}
    \bigg[\frac{\partial}{\partial x} \bigg(\Gamma \frac{\partial \phi}{\partial x} \bigg)\bigg]_{i} & \approx
    \frac{\bigg( \Gamma \frac{\partial\phi}{\partial x} \bigg)_{i+\frac{1}{2}} - \bigg( \Gamma \frac{\partial\phi}{\partial x} \bigg)_{i-\frac{1}{2}}}{\frac{1}{2}(x_{i+1}-x_{i-1})} \\
    & \approx \frac{\Gamma_{i+\frac{1}{2}}\frac{\phi_{i+1}-\phi_{i}}{x_{i+1}-x_{i}} - \Gamma_{i-\frac{1}{2}}\frac{\phi_{i}-\phi_{i-1}}{x_{i}-x_{i-1}}}{\frac{1}{2}(x_{i+1}-x_{i-1})}
\end{align*}
\end{frame}
%================================================
\begin{frame}{Approximation of the Mixed Derivatives}
Mixed derivatives only occur when the transport equations are expressed in non-orthogonal coordinate systems.
    \[
    \frac{\partial^{2} \phi}{\partial x \partial y} = \frac{\partial}{\partial x} \bigg( \frac{\partial \phi}{\partial y} \bigg)
    \]
This may be treated by combining the one-dimensional approximations for the second derivative.
\end{frame}
%================================================
\appendix
\begin{frame}{Reference}
 \bibliography{bibTeX/ferziger.bib}
 \bibliographystyle{abbrv}
\end{frame}
%================================================
\end{document}
